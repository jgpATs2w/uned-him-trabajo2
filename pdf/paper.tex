\documentclass[fleqn,10pt]{wlscirep}
\newcommand{\autocite}[1]{\cite{#1}}
\newcommand{\textcite}[1]{\cite{#1}}

\title{Trabajo 2}

%%------------AUTHORS--------------
\author[50201633Q / jgarcia1285 ]{Javier Garcia Parra}


%%------------/AUTHORS--------------


%\keywords{Keyword1, Keyword2, Keyword3}

\begin{abstract}
%% Text of abstract
Este es el segundo trabajo de la asignatura `Herramientas informáticas
para las matemáticas' del grado en matemáticas de la UNED, curso
2019/20.
\end{abstract}


\begin{document}

\flushbottom
\maketitle
% * <john.hammersley@gmail.com> 2015-02-09T12:07:31.197Z:
%
%  Click the title above to edit the author information and abstract
%
\thispagestyle{empty}

%\noindent Please note: Abbreviations should be introduced at the first mention in the main text – no abbreviations lists. Suggested structure of main text (not enforced) is provided below.

%% main text
\section{Introducción}\label{introducciuxf3n}

\section{Métodos}\label{muxe9todos}

Para la resolución se han empleado las herramientas maxima 5.41 y scilab
6.0.1.\\
Se ha empleado como base el código facilitado, empleando una única
función \texttt{procesa} tanto para el cifrado como para el descifrado.

\section{Resultados}\label{resultados}

2.1)

El código emplea un vector de alfabetos de cifrado que representan las
permutaciones asociadas a cada módulo base 3 (0, 1, 2). Para realizar el
cifrado recorre la cadena, caracter a caracter y va aplicando la
permutación correspondiente según el módulo base 3 de la posición del
caracter dentro de la cadena.

2.2) Se incluye el código solicitado en el archivo
\texttt{ejemplo\_cripto\_RGB\_T2.sce}.\\
Desde scilab o scilab-cli, ejecute
\texttt{exec(\textquotesingle{}ejemplo\_cripto\_RGB\_T2.sce)}

2.3) En el mismo archivo que el apartado anterior se incluye el cifrado
y descifrado requerido.

2.4) El código en máxima se encuentra en ejemplo\_cripto\_RGB\_T2.mac.
Desde maxima, ejecute
\texttt{batchload("ejemplo\_cripto\_RGB\_T2.mac");}

2.5) Se hace uso de la función Matplot. El código se encuentra en
\texttt{crea\_imagen.sce}.\\
Desde scilab o scilab-cli, ejecute
\texttt{exec(\textquotesingle{}ejemplo\_cripto\_RGB\_T2.sce)}





\end{document}
